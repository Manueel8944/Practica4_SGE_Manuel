\documentclass{article}

\usepackage[spanish]{babel}
\usepackage[letterpaper,top=2cm,bottom=2cm,left=3cm,right=3cm,marginparwidth=1.75cm]{geometry}
\usepackage{graphicx}
\usepackage[colorlinks=true, allcolors=blue]{hyperref}
\usepackage{graphics}
\usepackage{listings}
\usepackage{xcolor}
\usepackage{float}

\definecolor{codegray}{gray}{0.95}

\lstset{
    language=Python,
    backgroundcolor=\color{codegray},
    basicstyle=\ttfamily\footnotesize,
    breaklines=true,
    numbers=left,
    numbersep=8pt,
    frame=single,
    framerule=0.5pt,
    tabsize=2,
    inputencoding=utf8,
    extendedchars=true,
    literate={á}{{\'a}}1 {é}{{\'e}}1 {í}{{\'i}}1 {ó}{{\'o}}1 {ú}{{\'u}}1 {ñ}{{\~n}}1,
    commentstyle=\color{red}\itshape,
    morecomment=[s][\color{red}]{<!--}{-->},
}

\title{Arquitectura Aplicacion Movil}
\author{Manuel Añel García}

\begin{document}
\maketitle

\section{Introduction}

En este documento se va a explicar el trabajo de SGE que consta de 4 actividades en las que se van a modificar, crear y ampliar módulos de Odoo, para poder empezar con esta práctica hay que reutilizar el contenedor que teniamos ya de Odoo y clonar el repositorio de módulos dicho en los requisitos de la práctica.

\section{Actividad 01}
    
Primero tenemos que activar el modulo en el que vamos a trabajar, buscamos "lista de tareas" y activamos.
\begin{figure}[H]
    \centering
    \includegraphics[width=0.5\textwidth]{images/modulo-lista-tareas.png}
\end{figure}

Tenemos que entrar en addons al modulo EJ02-ListaTareas y vamos a su view.xml para hacer otra vista pero en formato kanban, para empezar tenemos que agregar kanban en el view mode como se ve a continuación.

\begin{figure}[H]
\begin{lstlisting}
<field name="view_mode">list,form,kanban</field>
\end{lstlisting}
\end{figure}

\begin{figure}[H] Hay que crear un nuevo record que va a pertenecer al formato kanaban, en los comentarios del código está explicado el funcionamiento 
\begin{lstlisting}
<record id="view_lista_tareas_kanban" model="ir.ui.view">
    <!-- Definimos el nombre interno, modelo odoo y arquitectura -->
    <field name="name">lista.tareas.kanban</field>
    <field name="model">lista_tareas.lista</field>
    <field name="arch" type="xml">

    <kanban>
        <!-- Definimos los campos del modelo -->
        <field name="tarea"/>
        <field name="prioridad"/>
        <field name="urgente"/>
        <field name="realizada"/>

        <templates>
            <!-- Creamos una caja kanban -->
            <t t-name="kanban-box">
                <div class="oe_kanban_global_click o_kanban_record">
                    
                    <!-- Título de la tarea -->
                    <div class="o_kanban_record_top">
                        <strong><t t-esc="record.tarea.value"/></strong>
                    </div>

                    <!-- Prioridad -->
                    <div>
                        Prioridad: <t t-esc="record.prioridad.value"/>
                    </div>

                    <!-- Checkbox realizada --> 
                    <div>
                        Realizada: <field name="realizada"/>
                    </div>

                    <!-- Checkbox urgente --> 
                    <div>
                        Urgente: <field name="urgente"/>
                    </div>

                </div>
            </t>
        </templates>
    </kanban>

    </field>
</record>
\end{lstlisting}
\end{figure}

Podemos ver como se ve la lista de tareas en formato kanban y como se puede alternar la vista arriba a la derecha
\begin{figure}[H]
    \centering
    \includegraphics[width=1\textwidth]{images/lista-tareas-kanban.png}
\end{figure}

\begin{figure}[H]Ahora hay que modificar las tareas para que tengan una fecha asignada y crear una nueva vista para visualizarlas en un formato Calendario según esa fecha. Empezamos poniendo en el archivo python un nuevo campo que va a ser fecha con formato Date.
\begin{lstlisting}
fecha = fields.Date(string="Fecha")
\end{lstlisting}
\end{figure}

\begin{figure}[H]Agregamos el campo fecha en la vista list y form
\begin{lstlisting}
<field name="fecha"/> 
\end{lstlisting}
\end{figure}

\begin{figure}[H]Donde antes pusimos kanban para para añadir el modo kanban hacemos lo mismo pero con calendar.
\begin{lstlisting}
<field name="view_mode">list,form,kanban,calendar</field>
\end{lstlisting}
\end{figure}

\begin{figure}[H]Ahora creamos el nuevo record para hacer la nueva vista
\begin{lstlisting}
<record id="view_lista_tareas_calendar" model="ir.ui.view">
    <!-- Definimos el nombre interno, modelo odoo y arquitectura -->
    <field name="name">lista.tareas.calendar</field>
    <field name="model">lista_tareas.lista</field>
    <field name="arch" type="xml">
        <!-- Creamos el calendario y decimos con el date_start que se coloque mediante el campo fecha -->
        <calendar string="Calendario de tareas" date_start="fecha">
            <!-- Se muestra el nombre de la tarea -->
            <field name="tarea"/>
        </calendar>
    </field>
</record>
\end{lstlisting}
\end{figure}

Vemos como se muestra correctamente en la fecha que puse
\begin{figure}[H]
    \centering
    \includegraphics[width=1\textwidth]{images/lista-tareas-calendario.png}
\end{figure}

\section{Actividad 02}
Primero hay que encontrar el modulo sobre el que vamos a trabajar que es y activarlo
\begin{figure}[H]
    \centering
    \includegraphics[width=0.5\textwidth]{images/modulo-biblioteca.png}
\end{figure}

\begin{figure}[H] Aqui lo que hacemos es crear un nuevo archivo .py para manejar los socios
\begin{lstlisting}
class BibliotecaSocio(models.Model):
    # Nombre interno del modelo en Odoo
    _name = 'biblioteca.socio'
    # Descripción del modelo
    _description = 'Socio de la biblioteca'

    # Campo para el nombre del socio, obligatorio
    nombre = fields.Char(string='Nombre', required=True)
    # Campo para el apellido del socio, obligatorio
    apellido = fields.Char(string='Apellido', required=True)
    # Campo para el identificador único del socio
    identificador = fields.Char(string='ID Socio', required=True, unique=True)

    prestamo_ids = fields.One2many( # Relación One2many: un socio puede tener varios ejemplares prestados
        'biblioteca.ejemplar', # Es el modelo relacionado
        'socio_id', #Es el campo en el modelo relacionado que apunta a este socio
        string='Ejemplares Prestados'
    )

    # Método para mostrar cómo se representa cada socio en campos Many2one
    def name_get(self):
        result = []
        for record in self:
            nombre_completo = f"{record.nombre} {record.apellido} ({record.identificador})"
            result.append((record.id, nombre_completo))
        return result
\end{lstlisting}
\end{figure}

\begin{figure}[H] Ahora haay que crear una nueva vista xml para el modelo que acabamos de crear
\begin{lstlisting}
<odoo>

    <!-- Acción que abre la vista de Socios -->
    <record id="biblioteca_socio_action" model="ir.actions.act_window">
        <field name="name">Socios</field> <!-- Nombre de la acción -->
        <field name="res_model">biblioteca.socio</field> <!-- Modelo que abre -->
        <field name="view_mode">list,form</field> <!-- Tipos de vista -->
        <field name="help" type="html">
            <p>
                Crea y gestiona los socios de la biblioteca.
            </p>
        </field>
    </record>

    <!-- Menú que apunta a la acción de Socios -->
    <menuitem name="Socios" id="menu_biblioteca_socio" parent="biblioteca_base_menu" action="biblioteca_socio_action"/>

    <!-- Vista lista de socios -->
    <record id="list" model="ir.ui.view">
        <field name="name">Lista de Socios</field>
        <field name="model">biblioteca.socio</field>
        <field name="arch" type="xml">
            <list>
                <field name="nombre"/> <!-- Muestra el nombre -->
                <field name="apellido"/> <!-- Muestra el apellido -->
                <field name="identificador"/> <!-- Muestra el ID del socio -->
            </list>
        </field>
    </record>

    <!-- Vista formulario de socio -->
    <record id="view_form_socio" model="ir.ui.view">
        <field name="name">Formulario Socio</field>
        <field name="model">biblioteca.socio</field>
        <field name="arch" type="xml">
            <form string="Socio de la Biblioteca">
                <group>
                    <field name="nombre"/> <!-- Campo nombre -->
                    <field name="apellido"/> <!-- Campo apellido -->
                    <field name="identificador"/> <!-- Campo ID socio -->
                    <field name="prestamo_ids" readonly="1"/> <!-- Relación con los ejemplares prestados, solo lectura -->
                </group>
            </form>
        </field>
    </record>

</odoo>
\end{lstlisting}
\end{figure}

\begin{figure}[H] Aqui podemos ver como se crean los socios
    \centering
    \includegraphics[width=1\textwidth]{images/creacion-socios.png}
\end{figure}

\begin{figure}[H] Aqui podemos ver un socio que tiene un ejemplar, esto ya lo veremos mas alante
    \centering
    \includegraphics[width=1\textwidth]{images/ver-socios.png}
\end{figure}

\begin{figure}[H] Ahora hay que crear un nuevo modelo para los ejemplares y tener en cuenta las fechas de prestamo.
\begin{lstlisting}
class BibliotecaEjemplar(models.Model):
    _name = 'biblioteca.ejemplar'  # Nombre técnico del modelo
    _description = 'Ejemplar de cómic prestable'  # Descripción del modelo

    # Campo Many2one que enlaza con el cómic al que pertenece este ejemplar (un comic puede tener varios ejemplares)
    comic_id = fields.Many2one(
        'biblioteca.comic',
        string='Cómic',  # Etiqueta que se muestra en la interfaz
        required=True,   # Campo obligatorio
        ondelete='cascade'  # Si se borra el cómic, se borran sus ejemplares
    )

    # Campo Many2one que enlaza con el socio que tiene prestado el ejemplar (un socio puede tener varios ejemplares)
    socio_id = fields.Many2one(
        'biblioteca.socio',
        string='Prestado a'  # Etiqueta en la interfaz
    )

    # Fecha en que se realizó el préstamo
    fecha_prestamo = fields.Date(string='Fecha de préstamo')
    # Fecha prevista para devolver el ejemplar
    fecha_devolucion_prevista = fields.Date(string='Fecha prevista de devolución')

    # Estado del ejemplar: disponible o prestado
    estado = fields.Selection(
        [('disponible', 'Disponible'),
         ('prestado', 'Prestado')],
        string='Estado',
        default='disponible'  # Valor por defecto
    )

    # La fecha de préstamo no puede ser futura
    @api.constrains('fecha_prestamo')
    def _check_fecha_prestamo(self):
        hoy = fields.Date.today()  # Fecha actual
        for record in self:
            if record.fecha_prestamo and record.fecha_prestamo > hoy:
                raise ValidationError(
                    'La fecha de préstamo no puede ser posterior al día actual.'
                )

    #La fecha prevista de devolución no puede ser anterior a hoy
    @api.constrains('fecha_devolucion_prevista')
    def _check_fecha_devolucion(self):
        hoy = fields.Date.today()  # Fecha actual
        for record in self:
            if record.fecha_devolucion_prevista and record.fecha_devolucion_prevista < hoy:
                raise ValidationError(
                    'La fecha prevista de devolución no puede ser anterior al día actual.'
                )

    # Cambio automático del estado cuando se selecciona o quita un socio
    @api.onchange('socio_id')
    def _onchange_socio(self):
        for record in self:
            if record.socio_id:  # Si hay un socio asignado
                record.estado = 'prestado'  # Cambia a prestado
            else:
                record.estado = 'disponible'  # Si no hay socio, está disponible
\end{lstlisting}
\end{figure}

\begin{figure}[H] Ahora hay que crear una nueva vista para el modelo que acabamos de crear
\begin{lstlisting}
<odoo>

    <!-- Acción para abrir la vista de Ejemplares (listado y formulario) -->
    <record id="biblioteca_ejemplar_action" model="ir.actions.act_window">
        <field name="name">Ejemplares de Cómics</field> <!-- Nombre que verá el usuario -->
        <field name="res_model">biblioteca.ejemplar</field> <!-- Modelo al que apunta -->
        <field name="view_mode">list,form</field> <!-- Vistas disponibles: lista y formulario -->
    </record>

    <!-- Menú que abre la acción de Ejemplares dentro del menú base "Mi biblioteca" -->
    <menuitem name="Ejemplares" 
              id="menu_biblioteca_ejemplar" 
              parent="biblioteca_base_menu" 
              action="biblioteca_ejemplar_action"/>

    <!-- Vista Formulario para crear o editar un ejemplar -->
    <record id="view_form_ejemplar" model="ir.ui.view">
        <field name="name">Formulario Ejemplar</field> <!-- Nombre interno de la vista -->
        <field name="model">biblioteca.ejemplar</field> <!-- Modelo asociado -->
        <field name="arch" type="xml">
            <form string="Ejemplar de Cómic">
                <group>
                    <field name="comic_id"/> <!-- Relación con el cómic correspondiente -->
                    <field name="socio_id"/> <!-- Relación con el socio que lo tiene prestado -->
                    <field name="estado" readonly="1"/> <!-- Estado del ejemplar (solo lectura) -->
                    <field name="fecha_prestamo"/> <!-- Fecha del préstamo -->
                    <field name="fecha_devolucion_prevista"/> <!-- Fecha prevista de devolución -->
                </group>
            </form>
        </field>
    </record>

    <!-- Vista Lista para ver todos los ejemplares en forma de tabla -->
    <record id="view_list_ejemplar" model="ir.ui.view">
        <field name="name">Lista Ejemplares</field> <!-- Nombre interno de la vista -->
        <field name="model">biblioteca.ejemplar</field> <!-- Modelo asociado -->
        <field name="arch" type="xml">
            <list>
                <field name="comic_id"/> <!-- Columna: Cómic -->
                <field name="socio_id"/> <!-- Columna: Socio -->
                <field name="estado"/> <!-- Columna: Estado -->
                <field name="fecha_prestamo"/> <!-- Columna: Fecha de préstamo -->
                <field name="fecha_devolucion_prevista"/> <!-- Columna: Fecha prevista de devolución -->
            </list>
        </field>
    </record>

</odoo>
\end{lstlisting}
\end{figure}
Aqui tuve un fallo porque hay que acordarse de los modelos nuevos meterlos aqui
\begin{figure}[H] 
    \centering
    \includegraphics[width=0.5\textwidth]{images/biblioteca-fallo-init.png}
\end{figure}
Y esto lo mismo que hay que acordarse de meter las vistas nuevas
\begin{figure}[H] 
    \centering
    \includegraphics[width=0.5\textwidth]{images/biblioteca-fallo-manifest.png}
\end{figure}

\section{Actividad 03}
En esta actividad hay que crear un nuevo módulo desde 0 en el que gestione un hospital con sus pacientes, médicos y consultas. Para empezar hay que crear una estructura simple de carpetas y archivos para nuestro módulo, como se ve a continuación.
\begin{figure}[H] 
    \centering
    \includegraphics[width=0.3\textwidth]{images/estructura-carpetas-hospital.png}
\end{figure}

Ahora hay que rellenar los archivos básicos del módulo para que funcione el modulo y este todo enlazado. Hay que tener muy en cuenta que las nuevas vistas hay que ponerlas en el manifest y los nuevos modelos en el init dentro de models.

\begin{figure}[H] 
    \centering
    \includegraphics[width=0.4\textwidth]{images/manifest-hospital.png}
\end{figure}

\begin{figure}[H] 
    \centering
    \includegraphics[width=0.3\textwidth]{images/init-hospital.png}
\end{figure}

\begin{figure}[H] 
    \centering
    \includegraphics[width=0.3\textwidth]{images/init-models-hospital.png}
\end{figure}

Tambien hay que tener en cuenta que hay que rellenar el archivo csv dentro de security para los permisos de los modelos

\begin{figure}[H] 
    \centering
    \includegraphics[width=0.8\textwidth]{images/security-hospital.png}
\end{figure}

Ahora vamos a crear los modelos y las vistas, la explicacion del código está en los comentarios del mismo

\begin{figure}[H] Modelo paciente
\begin{lstlisting}
class Paciente(models.Model):
_name = 'hospital.paciente'
_description = 'Paciente del hospital'

name = fields.Char('Nombre y Apellidos', required=True) # El char se diferencia del text de que tiene un limite de caracteres de 255 y text no
sintomas = fields.Text('Síntomas')

# Es una relacion varios a varios, un paciente puede tener varios medicos y viceversa
consulta_ids = fields.Many2many('hospital.medico', # Modelo destino (el otro modelo)
                                'consulta_paciente_medico', # Nombre de la tabla intermedia en la base de datos
                                'paciente_id', 'medico_id', # Columnas en la tabla intermedia
                                string='Médicos que lo atendieron')
\end{lstlisting}
\end{figure}

\begin{figure}[H] Modelo medico
\begin{lstlisting}
class Medico(models.Model):
    _name = 'hospital.medico'
    _description = 'Medico del hospital'

    name = fields.Char('Nombre y Apellidos', required=True)
    
    numero_colegiado = fields.Char('Número de colegiado')
    
    # Es una relacion varios a varios, un paciente puede tener varios medicos y viceversa
    consulta_ids = fields.Many2many('hospital.paciente', # Modelo destino (paciente)
                                    'consulta_paciente_medico', # Nombre de la tabla intermedia en la base de datos
                                    'medico_id', 'paciente_id', # Columnas en la tabla intermedia
                                    string='Pacientes atendidos')
\end{lstlisting}
\end{figure}

\begin{figure}[H] Modelo consulta , en este modelo es importante la funcion de crear los registros porque al hacer la consulta no se actualiza automatico en la relacion de paciente medico así que lo hacemos con la funcion
\begin{lstlisting}
class Consulta(models.Model):
    _name = 'hospital.consulta'
    _description = 'Consulta de un paciente con un médico'

    paciente_id = fields.Many2one('hospital.paciente', string='Paciente', required=True) # Relacion de que un paciente puede tener varias consultas y al reves no
    medico_id = fields.Many2one('hospital.medico', string='Médico', required=True) # Relacion de que un medico puede tener varias consultas y al reves no
    diagnostico = fields.Text('Diagnóstico') # Texto en el que se pone el diagnóstico
    fecha = fields.Datetime('Fecha', default=fields.Datetime.now) # Ponemos la fecha para el momento en el que se crea la consulta
    
    @api.model
    def create(self, vals):
        # Crear la consulta
        record = super().create(vals)
        # Actualizar Many2many del paciente-medico
        if record.paciente_id and record.medico_id: # Verifica si estan los valores llenos
            record.paciente_id.write({'consulta_ids': [(4, record.medico_id.id)]}) # Creamos los registros, el 4 es para añadir sin borrar nada
            record.medico_id.write({'consulta_ids': [(4, record.paciente_id.id)]})
        return record
\end{lstlisting}
\end{figure}

\begin{figure}[H] Vista paciente
\begin{lstlisting}
<odoo>
    <record id="view_form_paciente" model="ir.ui.view">
        <field name="name">paciente.form</field>
        <field name="model">hospital.paciente</field>
        <field name="arch" type="xml">
            <form string="Paciente"> <!-- Utilizamos el form para la creacion del paciente -->
                <sheet>
                    <group>
                        <field name="name"/> <!-- Eston son los atributos que se van a rellenar al crear el paciente -->
                        <field name="sintomas"/>
                        <field name="consulta_ids"/>
                    </group>
                </sheet>
            </form>
        </field>
    </record>

    <record id="view_list_paciente" model="ir.ui.view"> <!-- La lista se usa para la visualización de los pacientes como en una lista -->
        <field name="name">paciente.list</field>
        <field name="model">hospital.paciente</field>
        <field name="arch" type="xml">
            <list string="Pacientes">
                <field name="name"/> <!-- La informacion que se va a ver en la lista solamente va a ser el nombre -->
            </list>
        </field>
    </record>

    <record id="action_paciente" model="ir.actions.act_window"> <!-- Aqui se determina lo que va haber en la vista que es un form y una list -->
        <field name="name">Pacientes</field>
        <field name="res_model">hospital.paciente</field>
        <field name="view_mode">list,form</field>
    </record>

    <menuitem id="menu_hospital_root" name="Hospital"/> <!-- Aqui creamos básicamente el boton del menu de arriba para cambiar entre pacientes, medicos etc... -->
    <menuitem id="menu_paciente" name="Pacientes" parent="menu_hospital_root" action="action_paciente"/>
</odoo>
\end{lstlisting}
\end{figure}

\begin{figure}[H] Vista medico
\begin{lstlisting}
<odoo>
    <record id="view_form_medico" model="ir.ui.view">
        <field name="name">medico.form</field>
        <field name="model">hospital.medico</field>
        <field name="arch" type="xml">
            <form string="Médico"> <!-- Utilizamos el form para la creacion del medico -->
                <sheet>
                    <group>
                        <field name="name"/> <!-- Eston son los atributos que se van a rellenar al crear el medico -->
                        <field name="numero_colegiado"/>
                        <field name="consulta_ids"/>
                    </group>
                </sheet>
            </form>
        </field>
    </record>

    <record id="view_list_medico" model="ir.ui.view"> <!-- La lista se usa para la visualización de los medicos como en una lista -->
        <field name="name">medico.list</field>
        <field name="model">hospital.medico</field>
        <field name="arch" type="xml">
            <list string="Médicos">
                <field name="name"/> <!-- La informacion que se va a ver en la lista solamente va a ser el nombre y el numero de colegiado -->
                <field name="numero_colegiado"/>
            </list>
        </field>
    </record>

    <record id="action_medico" model="ir.actions.act_window"> <!-- Aqui se determina lo que va haber en la vista que es un form y una list -->
        <field name="name">Médicos</field>
        <field name="res_model">hospital.medico</field>
        <field name="view_mode">list,form</field>
    </record>

    <!-- Aqui creamos básicamente el boton del menu de arriba para cambiar entre pacientes, medicos etc... -->
    <menuitem id="menu_medico" name="Médicos" parent="menu_hospital_root" action="action_medico"/> 
</odoo>
\end{lstlisting}
\end{figure}

\begin{figure}[H] Vista consulta
\begin{lstlisting}
<odoo>
    <record id="view_form_consulta" model="ir.ui.view">
        <field name="name">consulta.form</field>
        <field name="model">hospital.consulta</field>
        <field name="arch" type="xml">
            <form string="Consulta"> <!-- Utilizamos el form para la creacion de la consulta -->
                <sheet>
                    <group>
                        <field name="paciente_id"/> <!-- Eston son los atributos que se van a rellenar al crear la consulta -->
                        <field name="medico_id"/>
                        <field name="diagnostico"/>
                        <field name="fecha"/> <!-- La fecha se pone sola -->
                    </group>
                </sheet>
            </form>
        </field>
    </record>

    <record id="view_list_consulta" model="ir.ui.view"> <!-- La lista se usa para la visualización de las consultas como en una lista -->
        <field name="name">consulta.list</field>
        <field name="model">hospital.consulta</field>
        <field name="arch" type="xml">
            <list string="Consultas">
                <field name="paciente_id"/> <!-- La informacion que se va a ver en la lista -->
                <field name="medico_id"/>
                <field name="fecha"/>
            </list>
        </field>
    </record>

    <record id="action_consulta" model="ir.actions.act_window"> <!-- Aqui se determina lo que va haber en la vista que es un form y una list -->
        <field name="name">Consultas</field>
        <field name="res_model">hospital.consulta</field>
        <field name="view_mode">list,form</field>
    </record>

    <!-- Aqui creamos básicamente el boton del menu de arriba para cambiar entre pacientes, medicos y consultas -->
    <menuitem id="menu_consulta" name="Consultas" parent="menu_hospital_root" action="action_consulta"/>
</odoo>
\end{lstlisting}
\end{figure}

\begin{figure}[H]Aqui vemos como crear un paciente
    \centering
    \includegraphics[width=1\textwidth]{images/crear-paciente-hospital.png}
\end{figure}

\begin{figure}[H]Aqui vemos como crear un medico
    \centering
    \includegraphics[width=1\textwidth]{images/crear-medico_hospital.png}
\end{figure}

\begin{figure}[H]Aqui vemos como crear una consulta con los pacientes y medicos creados gracias a las relaciones
    \centering
    \includegraphics[width=1\textwidth]{images/crear-consulta-hospital.png}
\end{figure}

\begin{figure}[H]Aqui vemos como se visualizan los pacientes desde la lista y el formulario, en el formulario se puede ver la relacion con el medico al haber creado la consulta gracias a la funcion definida
    \centering
    \includegraphics[width=1\textwidth]{images/ver-pacientes-hospital.png}
\end{figure}

\begin{figure}[H]
    \centering
    \includegraphics[width=1\textwidth]{images/ver-pacientes-dentro-hospital.png}
\end{figure}

\begin{figure}[H]Aqui vemos como se visualizan los medicos desde la lista y el formulario, en el formulario se puede ver la relacion con el paciente al haber creado la consulta gracias a la funcion definida
    \centering
    \includegraphics[width=1\textwidth]{images/ver-medicos-hospital.png}
\end{figure}

\begin{figure}[H]
    \centering
    \includegraphics[width=1\textwidth]{images/ver-medicos-dentro-hospital.png}
\end{figure}

\begin{figure}[H]Aqui podemos ver las consultas creadas
    \centering
    \includegraphics[width=1\textwidth]{images/ver-consultas-hospital.png}
\end{figure}

\section{Actividad 04}


\end{document}