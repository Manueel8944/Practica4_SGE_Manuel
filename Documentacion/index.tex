\documentclass{article}

\usepackage[spanish]{babel}
\usepackage[letterpaper,top=2cm,bottom=2cm,left=3cm,right=3cm,marginparwidth=1.75cm]{geometry}
\usepackage{graphicx}
\usepackage[colorlinks=true, allcolors=blue]{hyperref}
\usepackage{graphics}
\usepackage{listings}
\usepackage{xcolor}
\usepackage{float}

\definecolor{codegray}{gray}{0.95}

\lstset{
    backgroundcolor=\color{codegray},
    basicstyle=\ttfamily\footnotesize,
    breaklines=true,
    numbers=left,
    numbersep=8pt,
    frame=single,
    framerule=0.5pt,
    tabsize=2,
    inputencoding=utf8,
    extendedchars=true,
    literate={á}{{\'a}}1 {é}{{\'e}}1 {í}{{\'i}}1 {ó}{{\'o}}1 {ú}{{\'u}}1 {ñ}{{\~n}}1
}

\title{Arquitectura Aplicacion Movil}
\author{Manuel Añel García}

\begin{document}
\maketitle

\section{Introduction}

En este documento se va a explicar el trabajo de SGE que consta de 4 actividades en las que se van a modificar, crear y ampliar módulos de Odoo, para poder empezar con esta práctica hay que reutilizar el contenedor que teniamos ya de Odoo y clonar el repositorio de módulos dicho en los requisitos de la práctica.

\section{Actividad 01}
    
Primero tenemos que activar el modulo en el que vamos a trabajar, buscamos "lista de tareas" y activamos.
\begin{figure}[H]
    \centering
    \includegraphics[width=0.5\textwidth]{images/modulo-lista-tareas.png}
\end{figure}

Tenemos que entrar en addons al modulo EJ02-ListaTareas y vamos a su view.xml para hacer otra vista pero en formato kanban, para empezar tenemos que agregar kanban en el view mode como se ve a continuación.

\begin{figure}[H]
\begin{lstlisting}
<field name="view_mode">list,form,kanban</field>
\end{lstlisting}
\end{figure}

\begin{figure}[H] Hay que crear un nuevo record que va a pertenecer al formato kanaban, en los comentarios del código está explicado el funcionamiento 
\begin{lstlisting}
<record id="view_lista_tareas_kanban" model="ir.ui.view">
    <!-- Definimos el nombre interno, modelo odoo y arquitectura -->
    <field name="name">lista.tareas.kanban</field>
    <field name="model">lista_tareas.lista</field>
    <field name="arch" type="xml">

    <kanban>
        <!-- Definimos los campos del modelo -->
        <field name="tarea"/>
        <field name="prioridad"/>
        <field name="urgente"/>
        <field name="realizada"/>

        <templates>
            <!-- Creamos una caja kanban -->
            <t t-name="kanban-box">
                <div class="oe_kanban_global_click o_kanban_record">
                    
                    <!-- Título de la tarea -->
                    <div class="o_kanban_record_top">
                        <strong><t t-esc="record.tarea.value"/></strong>
                    </div>

                    <!-- Prioridad -->
                    <div>
                        Prioridad: <t t-esc="record.prioridad.value"/>
                    </div>

                    <!-- Checkbox realizada --> 
                    <div>
                        Realizada: <field name="realizada"/>
                    </div>

                    <!-- Checkbox urgente --> 
                    <div>
                        Urgente: <field name="urgente"/>
                    </div>

                </div>
            </t>
        </templates>
    </kanban>

    </field>
</record>
\end{lstlisting}
\end{figure}

Podemos ver como se ve la lista de tareas en formato kanban y como se puede alternar la vista arriba a la derecha
\begin{figure}[H]
    \centering
    \includegraphics[width=1\textwidth]{images/lista-tareas-kanban.png}
\end{figure}

\begin{figure}[H]Ahora hay que modificar las tareas para que tengan una fecha asignada y crear una nueva vista para visualizarlas en un formato Calendario según esa fecha. Empezamos poniendo en el archivo python un nuevo campo que va a ser fecha con formato Date.
\begin{lstlisting}
fecha = fields.Date(string="Fecha")
\end{lstlisting}
\end{figure}

\begin{figure}[H]Agregamos el campo fecha en la vista list y form
\begin{lstlisting}
<field name="fecha"/> 
\end{lstlisting}
\end{figure}

\begin{figure}[H]Donde antes pusimos kanban para para añadir el modo kanban hacemos lo mismo pero con calendar.
\begin{lstlisting}
<field name="view_mode">list,form,kanban,calendar</field>
\end{lstlisting}
\end{figure}

\begin{figure}[H]Ahora creamos el nuevo record para hacer la nueva vista
\begin{lstlisting}
<record id="view_lista_tareas_calendar" model="ir.ui.view">
    <!-- Definimos el nombre interno, modelo odoo y arquitectura -->
    <field name="name">lista.tareas.calendar</field>
    <field name="model">lista_tareas.lista</field>
    <field name="arch" type="xml">
        <!-- Creamos el calendario y decimos con el date_start que se coloque mediante el campo fecha -->
        <calendar string="Calendario de tareas" date_start="fecha">
            <!-- Se muestra el nombre de la tarea -->
            <field name="tarea"/>
        </calendar>
    </field>
</record>
\end{lstlisting}
\end{figure}

Vemos como se muestra correctamente en la fecha que puse
\begin{figure}[H]
    \centering
    \includegraphics[width=1\textwidth]{images/lista-tareas-calendario.png}
\end{figure}


\end{document}